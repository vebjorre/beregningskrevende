\documentclass[]{article}
\usepackage{lmodern}
\usepackage{amssymb,amsmath}
\usepackage{ifxetex,ifluatex}
\usepackage{fixltx2e} % provides \textsubscript
\ifnum 0\ifxetex 1\fi\ifluatex 1\fi=0 % if pdftex
  \usepackage[T1]{fontenc}
  \usepackage[utf8]{inputenc}
\else % if luatex or xelatex
  \ifxetex
    \usepackage{mathspec}
  \else
    \usepackage{fontspec}
  \fi
  \defaultfontfeatures{Ligatures=TeX,Scale=MatchLowercase}
\fi
% use upquote if available, for straight quotes in verbatim environments
\IfFileExists{upquote.sty}{\usepackage{upquote}}{}
% use microtype if available
\IfFileExists{microtype.sty}{%
\usepackage{microtype}
\UseMicrotypeSet[protrusion]{basicmath} % disable protrusion for tt fonts
}{}
\usepackage[margin=1in]{geometry}
\usepackage{hyperref}
\hypersetup{unicode=true,
            pdftitle={Computer Intensive Statistical Methods-Exercise 1},
            pdfauthor={Vebjørn Rekkebo, Camilla Karlsen},
            pdfborder={0 0 0},
            breaklinks=true}
\urlstyle{same}  % don't use monospace font for urls
\usepackage{color}
\usepackage{fancyvrb}
\newcommand{\VerbBar}{|}
\newcommand{\VERB}{\Verb[commandchars=\\\{\}]}
\DefineVerbatimEnvironment{Highlighting}{Verbatim}{commandchars=\\\{\}}
% Add ',fontsize=\small' for more characters per line
\usepackage{framed}
\definecolor{shadecolor}{RGB}{248,248,248}
\newenvironment{Shaded}{\begin{snugshade}}{\end{snugshade}}
\newcommand{\AlertTok}[1]{\textcolor[rgb]{0.94,0.16,0.16}{#1}}
\newcommand{\AnnotationTok}[1]{\textcolor[rgb]{0.56,0.35,0.01}{\textbf{\textit{#1}}}}
\newcommand{\AttributeTok}[1]{\textcolor[rgb]{0.77,0.63,0.00}{#1}}
\newcommand{\BaseNTok}[1]{\textcolor[rgb]{0.00,0.00,0.81}{#1}}
\newcommand{\BuiltInTok}[1]{#1}
\newcommand{\CharTok}[1]{\textcolor[rgb]{0.31,0.60,0.02}{#1}}
\newcommand{\CommentTok}[1]{\textcolor[rgb]{0.56,0.35,0.01}{\textit{#1}}}
\newcommand{\CommentVarTok}[1]{\textcolor[rgb]{0.56,0.35,0.01}{\textbf{\textit{#1}}}}
\newcommand{\ConstantTok}[1]{\textcolor[rgb]{0.00,0.00,0.00}{#1}}
\newcommand{\ControlFlowTok}[1]{\textcolor[rgb]{0.13,0.29,0.53}{\textbf{#1}}}
\newcommand{\DataTypeTok}[1]{\textcolor[rgb]{0.13,0.29,0.53}{#1}}
\newcommand{\DecValTok}[1]{\textcolor[rgb]{0.00,0.00,0.81}{#1}}
\newcommand{\DocumentationTok}[1]{\textcolor[rgb]{0.56,0.35,0.01}{\textbf{\textit{#1}}}}
\newcommand{\ErrorTok}[1]{\textcolor[rgb]{0.64,0.00,0.00}{\textbf{#1}}}
\newcommand{\ExtensionTok}[1]{#1}
\newcommand{\FloatTok}[1]{\textcolor[rgb]{0.00,0.00,0.81}{#1}}
\newcommand{\FunctionTok}[1]{\textcolor[rgb]{0.00,0.00,0.00}{#1}}
\newcommand{\ImportTok}[1]{#1}
\newcommand{\InformationTok}[1]{\textcolor[rgb]{0.56,0.35,0.01}{\textbf{\textit{#1}}}}
\newcommand{\KeywordTok}[1]{\textcolor[rgb]{0.13,0.29,0.53}{\textbf{#1}}}
\newcommand{\NormalTok}[1]{#1}
\newcommand{\OperatorTok}[1]{\textcolor[rgb]{0.81,0.36,0.00}{\textbf{#1}}}
\newcommand{\OtherTok}[1]{\textcolor[rgb]{0.56,0.35,0.01}{#1}}
\newcommand{\PreprocessorTok}[1]{\textcolor[rgb]{0.56,0.35,0.01}{\textit{#1}}}
\newcommand{\RegionMarkerTok}[1]{#1}
\newcommand{\SpecialCharTok}[1]{\textcolor[rgb]{0.00,0.00,0.00}{#1}}
\newcommand{\SpecialStringTok}[1]{\textcolor[rgb]{0.31,0.60,0.02}{#1}}
\newcommand{\StringTok}[1]{\textcolor[rgb]{0.31,0.60,0.02}{#1}}
\newcommand{\VariableTok}[1]{\textcolor[rgb]{0.00,0.00,0.00}{#1}}
\newcommand{\VerbatimStringTok}[1]{\textcolor[rgb]{0.31,0.60,0.02}{#1}}
\newcommand{\WarningTok}[1]{\textcolor[rgb]{0.56,0.35,0.01}{\textbf{\textit{#1}}}}
\usepackage{graphicx,grffile}
\makeatletter
\def\maxwidth{\ifdim\Gin@nat@width>\linewidth\linewidth\else\Gin@nat@width\fi}
\def\maxheight{\ifdim\Gin@nat@height>\textheight\textheight\else\Gin@nat@height\fi}
\makeatother
% Scale images if necessary, so that they will not overflow the page
% margins by default, and it is still possible to overwrite the defaults
% using explicit options in \includegraphics[width, height, ...]{}
\setkeys{Gin}{width=\maxwidth,height=\maxheight,keepaspectratio}
\IfFileExists{parskip.sty}{%
\usepackage{parskip}
}{% else
\setlength{\parindent}{0pt}
\setlength{\parskip}{6pt plus 2pt minus 1pt}
}
\setlength{\emergencystretch}{3em}  % prevent overfull lines
\providecommand{\tightlist}{%
  \setlength{\itemsep}{0pt}\setlength{\parskip}{0pt}}
\setcounter{secnumdepth}{0}
% Redefines (sub)paragraphs to behave more like sections
\ifx\paragraph\undefined\else
\let\oldparagraph\paragraph
\renewcommand{\paragraph}[1]{\oldparagraph{#1}\mbox{}}
\fi
\ifx\subparagraph\undefined\else
\let\oldsubparagraph\subparagraph
\renewcommand{\subparagraph}[1]{\oldsubparagraph{#1}\mbox{}}
\fi

%%% Use protect on footnotes to avoid problems with footnotes in titles
\let\rmarkdownfootnote\footnote%
\def\footnote{\protect\rmarkdownfootnote}

%%% Change title format to be more compact
\usepackage{titling}

% Create subtitle command for use in maketitle
\providecommand{\subtitle}[1]{
  \posttitle{
    \begin{center}\large#1\end{center}
    }
}

\setlength{\droptitle}{-2em}

  \title{Computer Intensive Statistical Methods-Exercise 1}
    \pretitle{\vspace{\droptitle}\centering\huge}
  \posttitle{\par}
    \author{Vebjørn Rekkebo, Camilla Karlsen}
    \preauthor{\centering\large\emph}
  \postauthor{\par}
      \predate{\centering\large\emph}
  \postdate{\par}
    \date{04.02.2020}


\begin{document}
\maketitle

\hypertarget{problem-a-stochastic-simulation-by-the-probability-integral-transform-and-bivariate-techniques}{%
\section{Problem A: Stochastic simulation by the probability integral
transform and bivariate
techniques}\label{problem-a-stochastic-simulation-by-the-probability-integral-transform-and-bivariate-techniques}}

The inverse of the exponential cumulative density function with rate
parameter \(\lambda\) is given by \[
F^{-1}(u) = -\frac{1}{\lambda}\text{log}(u).
\] Hence we can sample from the exponential distribution by generating
\(u\sim \mathcal{U}[0,1]\) and returning
\(-\frac{1}{\lambda}\text{log}(u)\).

\begin{Shaded}
\begin{Highlighting}[]
\CommentTok{#This function returns a vector of n independent samples from the exponential distribution with rate parameter lam, generated from the inverse cumulative distribution. }
\NormalTok{random.exp <-}\StringTok{ }\ControlFlowTok{function}\NormalTok{(n,lam)\{}
  \KeywordTok{return}\NormalTok{(}\OperatorTok{-}\DecValTok{1}\OperatorTok{/}\NormalTok{lam }\OperatorTok{*}\StringTok{ }\KeywordTok{log}\NormalTok{(}\KeywordTok{runif}\NormalTok{(n)))}
\NormalTok{\}}
\end{Highlighting}
\end{Shaded}

We know that \(\text{E}(X)=\frac{1}{\lambda}\) and
\(\text{Var}(X)=\frac{1}{\lambda^2}\). The empirical mean is given by
the average of the sample and the empirical variance is
\(\widehat{\text{Var}(X)}=\frac{\sum_{i=1}^n x_i}{n-1}\).

\begin{Shaded}
\begin{Highlighting}[]
\CommentTok{#Testing function}
\NormalTok{test.function <-}\StringTok{ }\ControlFlowTok{function}\NormalTok{(n,func,arg,amean,avar)\{}
\NormalTok{  exs.sample=}\KeywordTok{func}\NormalTok{(n,arg)}
  \KeywordTok{cat}\NormalTok{(}\StringTok{"Empirical mean:"}\NormalTok{, }\KeywordTok{mean}\NormalTok{(exs.sample), }\StringTok{"Analytical mean:"}\NormalTok{,amean, }\StringTok{"}\CharTok{\textbackslash{}n}\StringTok{"}\NormalTok{)}
  \KeywordTok{cat}\NormalTok{(}\StringTok{"Empirical variance:"}\NormalTok{, }\KeywordTok{var}\NormalTok{(exs.sample), }\StringTok{"Analytical variance:"}\NormalTok{,avar,}\StringTok{"}\CharTok{\textbackslash{}n}\StringTok{"}\NormalTok{)}
\NormalTok{\}}

\CommentTok{#Make histogram of sample}
\NormalTok{lam <-}\StringTok{ }\DecValTok{3}
\NormalTok{sample.exp <-}\StringTok{ }\KeywordTok{enframe}\NormalTok{(}\KeywordTok{random.exp}\NormalTok{(}\DecValTok{100000}\NormalTok{,lam))}

\KeywordTok{ggplot}\NormalTok{() }\OperatorTok{+}
\StringTok{  }\KeywordTok{geom_histogram}\NormalTok{(}
    \DataTypeTok{data =} \KeywordTok{data.frame}\NormalTok{(sample.exp),}
    \DataTypeTok{mapping =} \KeywordTok{aes}\NormalTok{(}\DataTypeTok{x=}\NormalTok{value, }\DataTypeTok{y=}\NormalTok{..density..),}
    \DataTypeTok{binwidth =} \FloatTok{0.01}
\NormalTok{  ) }\OperatorTok{+}
\StringTok{  }\KeywordTok{geom_vline}\NormalTok{(}
    \KeywordTok{aes}\NormalTok{(}\DataTypeTok{xintercept=}\KeywordTok{mean}\NormalTok{(sample.exp}\OperatorTok{$}\NormalTok{value),}
        \DataTypeTok{col=}\StringTok{'Empirical mean'}\NormalTok{)}
\NormalTok{  ) }\OperatorTok{+}
\StringTok{  }\KeywordTok{stat_function}\NormalTok{(}
    \DataTypeTok{fun =}\NormalTok{ dexp,}
    \DataTypeTok{args =} \KeywordTok{list}\NormalTok{(}\DataTypeTok{rate=}\NormalTok{lam),}
    \KeywordTok{aes}\NormalTok{(}\DataTypeTok{col=}\StringTok{'Theoretical density'}\NormalTok{)}
\NormalTok{  ) }\OperatorTok{+}
\StringTok{  }\KeywordTok{labs}\NormalTok{(}
    \DataTypeTok{caption =} \KeywordTok{expression}\NormalTok{(}\StringTok{"Histogram of sample from exponential distribution with "}\OperatorTok{*}\NormalTok{lambda}\OperatorTok{*}\StringTok{"=3 compared to theoretical density."}\NormalTok{)}
\NormalTok{  )}
\end{Highlighting}
\end{Shaded}

\includegraphics{Project1_files/figure-latex/unnamed-chunk-3-1.pdf}

The histogram fits very well with the theoretical density function and
the mean is close to \(\frac{1}{\lambda}=\frac{1}{3}\) as it should be.

\begin{Shaded}
\begin{Highlighting}[]
\KeywordTok{test.function}\NormalTok{(}\DecValTok{10000}\NormalTok{,random.exp,lam,}\DecValTok{1}\OperatorTok{/}\NormalTok{lam,}\DecValTok{1}\OperatorTok{/}\NormalTok{lam}\OperatorTok{^}\DecValTok{2}\NormalTok{)}
\end{Highlighting}
\end{Shaded}

\begin{verbatim}
## Empirical mean: 0.3320791 Analytical mean: 0.3333333 
## Empirical variance: 0.1085954 Analytical variance: 0.1111111
\end{verbatim}

The printout shows that the empirical mean and variance are close to the
analytical values so the implementation seems correct.

We now consider the density function \[
g(x) = \begin{cases} cx^{\alpha-1}, &0<x<1,\\ ce^{-x}, &1\leq x, \\ 0, &\text{otherwise,} \end{cases}
\] where c is a normalising constant and \(\alpha \in (0,1)\).
Integration over the different intervals yields the cumulative
distribution \[
G(x) = \begin{cases} \frac{c}{\alpha}x^\alpha, &0<x<1,\\ c(\alpha^{-1}+e^{-1}-e^{-x}) &1\leq x, \\ 0, &\text{otherwise}. \end{cases}
\] We can now set \(u=G(x)\) and solve for \(x\) \[
\begin{cases} 
u=\frac{c}{\alpha}x^\alpha &\iff
x = (\frac{\alpha}{c}u)^{\frac{1}{\alpha}} , &0 < x < 1, \\
u = c(\alpha^{-1}+e^{-1}-e^{-x}) &\iff x=-\text{log}(\alpha^{-1}+e^{-1}-\frac{u}{c}), &1\leq x,
\end{cases}
\] and the boundary \(x=1\) is equivalent to \(u=\frac{c}{\alpha}\).
Hence the inverse of the cumulative distribution is \[
G^{-1}(u) = 
\begin{cases}
(\frac{\alpha}{c}u)^{\frac{1}{\alpha}} , &0 < u < \frac{c}{\alpha}, \\
-\text{log}(\alpha^{-1}+e^{-1}-e^{-x}), &\frac{c}{\alpha} \leq u, \\
0, &\text{otherwise}.
\end{cases}
\] The normalising constant \(c\) is found by integrating \(g(x)\) over
its domain and let the integral equal 1. \[
\int_0^\infty g(x)dx = c(\alpha^{-1}+e^{-1})=1 \iff c =\frac{\alpha e}{\alpha+e}.
\] Now we can sample from \(g(x)\) by generating
\(u\sim \mathcal{U}[0,1]\) and returning \(G^{-1}(u)\).

\begin{Shaded}
\begin{Highlighting}[]
\CommentTok{#Function for normalising constant in g distribution given parameter a (=alpha).}
\NormalTok{c.g <-}\StringTok{ }\ControlFlowTok{function}\NormalTok{(a)\{}
  \KeywordTok{return}\NormalTok{ (a}\OperatorTok{*}\KeywordTok{exp}\NormalTok{(}\DecValTok{1}\NormalTok{)}\OperatorTok{/}\NormalTok{(a}\OperatorTok{+}\KeywordTok{exp}\NormalTok{(}\DecValTok{1}\NormalTok{)))}
\NormalTok{\}}

\CommentTok{#Normalised density function for g (for plot)}
\NormalTok{g <-}\StringTok{ }\ControlFlowTok{function}\NormalTok{(x, a) \{}
\NormalTok{  c <-}\StringTok{ }\KeywordTok{c.g}\NormalTok{(a)}
\NormalTok{  out <-}\StringTok{ }\KeywordTok{rep}\NormalTok{(}\DecValTok{0}\NormalTok{, }\KeywordTok{length}\NormalTok{(x))}
\NormalTok{  left <-}\StringTok{ }\DecValTok{0} \OperatorTok{<}\StringTok{ }\NormalTok{x }\OperatorTok{&}\StringTok{ }\NormalTok{x }\OperatorTok{<}\StringTok{ }\DecValTok{1}
\NormalTok{  right <-}\StringTok{ }\DecValTok{1} \OperatorTok{<=}\StringTok{ }\NormalTok{x}
\NormalTok{  out[left] <-}\StringTok{ }\NormalTok{c }\OperatorTok{*}\StringTok{ }\NormalTok{(x[left] }\OperatorTok{^}\StringTok{ }\NormalTok{(a }\OperatorTok{-}\StringTok{ }\DecValTok{1}\NormalTok{))}
\NormalTok{  out[right] <-}\StringTok{ }\NormalTok{c }\OperatorTok{*}\StringTok{ }\KeywordTok{exp}\NormalTok{(}\OperatorTok{-}\NormalTok{x[right])}
\KeywordTok{return}\NormalTok{(out)}
\NormalTok{\}}

\CommentTok{#This function returns a vector of n independent samples from the g distribution with parameter a (=alpha), generated from the inverse cumulative distribution. }
\NormalTok{random.g <-}\StringTok{ }\ControlFlowTok{function}\NormalTok{(n,a)\{}
\NormalTok{  c <-}\StringTok{ }\KeywordTok{c.g}\NormalTok{(a) }\CommentTok{#Normalising constant}
\NormalTok{  u <-}\StringTok{ }\KeywordTok{runif}\NormalTok{(n)}
\NormalTok{  x <-}\StringTok{ }\KeywordTok{rep}\NormalTok{(}\DecValTok{0}\NormalTok{,n)}
  \ControlFlowTok{for}\NormalTok{ (i }\ControlFlowTok{in} \DecValTok{1}\OperatorTok{:}\NormalTok{n)\{}
    \ControlFlowTok{if}\NormalTok{ (u[i]}\OperatorTok{<}\NormalTok{c}\OperatorTok{/}\NormalTok{a)\{}
\NormalTok{      x[i] <-}\StringTok{ }\NormalTok{(a}\OperatorTok{*}\NormalTok{u[i]}\OperatorTok{/}\NormalTok{c)}\OperatorTok{^}\NormalTok{(}\DecValTok{1}\OperatorTok{/}\NormalTok{a)}
\NormalTok{    \}}
    \ControlFlowTok{else}\NormalTok{\{}
\NormalTok{      x[i] <-}\StringTok{ }\NormalTok{(}\OperatorTok{-}\KeywordTok{log}\NormalTok{(}\DecValTok{1}\OperatorTok{/}\NormalTok{a }\OperatorTok{+}\StringTok{ }\KeywordTok{exp}\NormalTok{(}\OperatorTok{-}\DecValTok{1}\NormalTok{) }\OperatorTok{-}\StringTok{ }\NormalTok{u[i]}\OperatorTok{/}\NormalTok{c))}
\NormalTok{    \}}
\NormalTok{  \}}
  \KeywordTok{return}\NormalTok{(x)}
\NormalTok{\}}
\end{Highlighting}
\end{Shaded}

The analytical mean and variance of \(X\sim g(x)\) are computed in the
usual way \begin{align}
\text{E}(X) &=\int_0^\infty xg(x)dx=\frac{c}{\alpha+1}+\frac{2c}{e}, \\
\text{E}(X^2) &=\int_0^\infty x^2g(x)dx=\frac{c}{\alpha+2}+\frac{5c}{e}, \\
\text{Var}(X) &=\text{E}(X^2)-(\text{E}(X))^2 = c(\frac{1}{\alpha+2}+\frac{5}{e})-(\frac{c}{\alpha+1}+\frac{2c}{e})^2. 
\end{align}

\begin{Shaded}
\begin{Highlighting}[]
\CommentTok{#Testing funtion}
\NormalTok{a <-}\StringTok{ }\FloatTok{.6} \CommentTok{#parameter}
\NormalTok{c <-}\StringTok{ }\KeywordTok{c.g}\NormalTok{(a) }\CommentTok{#normalising constant}
\NormalTok{g.mean <-}\StringTok{ }\NormalTok{c}\OperatorTok{/}\NormalTok{(a}\OperatorTok{+}\DecValTok{1}\NormalTok{)}\OperatorTok{+}\DecValTok{2}\OperatorTok{*}\NormalTok{c}\OperatorTok{*}\KeywordTok{exp}\NormalTok{(}\OperatorTok{-}\DecValTok{1}\NormalTok{) }\CommentTok{#analytical mean}
\NormalTok{g.var <-}\StringTok{ }\NormalTok{c}\OperatorTok{*}\NormalTok{(}\DecValTok{1}\OperatorTok{/}\NormalTok{(a}\OperatorTok{+}\DecValTok{2}\NormalTok{)}\OperatorTok{+}\StringTok{ }\DecValTok{5}\OperatorTok{*}\KeywordTok{exp}\NormalTok{(}\OperatorTok{-}\DecValTok{1}\NormalTok{)) }\OperatorTok{-}\StringTok{ }\NormalTok{g.mean}\OperatorTok{^}\DecValTok{2} \CommentTok{#analytical variance}
\KeywordTok{test.function}\NormalTok{(}\DecValTok{10000}\NormalTok{,random.g,a,g.mean,g.var)}
\end{Highlighting}
\end{Shaded}

\begin{verbatim}
## Empirical mean: 0.6743793 Analytical mean: 0.6688268 
## Empirical variance: 0.6433624 Analytical variance: 0.6457955
\end{verbatim}

The mean and variance of the sample is close to the theoretical values.

\begin{Shaded}
\begin{Highlighting}[]
\CommentTok{#Making histogram of sample from g}
\NormalTok{sample.g <-}\StringTok{ }\KeywordTok{enframe}\NormalTok{(}\KeywordTok{random.g}\NormalTok{(}\DecValTok{100000}\NormalTok{,a))}

\KeywordTok{ggplot}\NormalTok{() }\OperatorTok{+}
\StringTok{  }\KeywordTok{geom_histogram}\NormalTok{(}
    \DataTypeTok{data =} \KeywordTok{data.frame}\NormalTok{(sample.g),}
    \DataTypeTok{mapping =} \KeywordTok{aes}\NormalTok{(}\DataTypeTok{x=}\NormalTok{value, }\DataTypeTok{y=}\NormalTok{..density..),}
    \DataTypeTok{binwidth =} \FloatTok{0.01}\NormalTok{,}
    \DataTypeTok{na.rm =} \OtherTok{TRUE}
\NormalTok{  ) }\OperatorTok{+}
\StringTok{  }\KeywordTok{geom_vline}\NormalTok{(}
    \KeywordTok{aes}\NormalTok{(}\DataTypeTok{xintercept=}\KeywordTok{mean}\NormalTok{(sample.g}\OperatorTok{$}\NormalTok{value),}
        \DataTypeTok{col=}\StringTok{'Empirical mean'}\NormalTok{)}
\NormalTok{  ) }\OperatorTok{+}
\StringTok{  }\KeywordTok{stat_function}\NormalTok{(}
    \DataTypeTok{fun =}\NormalTok{ g,}
    \DataTypeTok{args =} \KeywordTok{list}\NormalTok{(}\DataTypeTok{a=}\NormalTok{a),}
    \KeywordTok{aes}\NormalTok{(}\DataTypeTok{col=}\StringTok{'Theoretical density'}\NormalTok{)}
\NormalTok{  ) }\OperatorTok{+}
\StringTok{  }\KeywordTok{ylim}\NormalTok{(}\DecValTok{0}\NormalTok{,}\DecValTok{2}\NormalTok{) }\OperatorTok{+}
\StringTok{  }\KeywordTok{xlim}\NormalTok{(}\DecValTok{0}\NormalTok{,}\DecValTok{5}\NormalTok{) }\OperatorTok{+}
\StringTok{    }\KeywordTok{labs}\NormalTok{(}
      \DataTypeTok{caption =} \KeywordTok{bquote}\NormalTok{(}\StringTok{"Histogram of sample from g-distribution with "}\OperatorTok{~}\NormalTok{alpha}\OperatorTok{*}\StringTok{"=0.6 compared to theoretical density."}\NormalTok{)}
\NormalTok{    )}
\end{Highlighting}
\end{Shaded}

\includegraphics{Project1_files/figure-latex/unnamed-chunk-7-1.pdf}

The histogram and density function coincide and that further indicates
that the implementation is correct.

The Box-Müller algorithm samples from the bivariate standard normal
distribution. We first generate \(x_1 \sim \mathcal{U}(0,2\pi)\) and
\(x_2 \sim \text{exp}(\frac{1}{2})\) and then return
\(y_1=\sqrt{x_2}\text{cos}(x_1)\) and \(y_2=\sqrt{x_2}\text{sin}(x_1)\).
We want to generate a vector of \(n\) independent samples from the
standard normal distribution. Thus we only keep \(y_1\) from each
bivariate sample.

\begin{Shaded}
\begin{Highlighting}[]
\CommentTok{#This function returns a vector of n independent samples from the standard normal distribution, generated by the Box-Müller algorithm}
\NormalTok{random.norm <-}\StringTok{ }\ControlFlowTok{function}\NormalTok{(n)\{}
\NormalTok{  x1 <-}\StringTok{ }\KeywordTok{runif}\NormalTok{(n)}\OperatorTok{*}\DecValTok{2}\OperatorTok{*}\NormalTok{pi}
\NormalTok{  x2 <-}\StringTok{ }\KeywordTok{random.exp}\NormalTok{(n,.}\DecValTok{5}\NormalTok{)}
  \KeywordTok{return}\NormalTok{(}\KeywordTok{sqrt}\NormalTok{(x2)}\OperatorTok{*}\KeywordTok{cos}\NormalTok{(x1))}
\NormalTok{\}}

\CommentTok{#Testing function}
\NormalTok{sample.normal=}\KeywordTok{enframe}\NormalTok{(}\KeywordTok{random.norm}\NormalTok{(}\DecValTok{100000}\NormalTok{))}
\KeywordTok{cat}\NormalTok{(}\StringTok{"Empirical mean:"}\NormalTok{, }\KeywordTok{mean}\NormalTok{(sample.normal), }\StringTok{"Analytical mean:"}\NormalTok{,}\FloatTok{0.0}\NormalTok{, }\StringTok{"}\CharTok{\textbackslash{}n}\StringTok{"}\NormalTok{)}
\end{Highlighting}
\end{Shaded}

\begin{verbatim}
## Warning in mean.default(sample.normal): argument is not numeric or logical:
## returning NA
\end{verbatim}

\begin{verbatim}
## Empirical mean: NA Analytical mean: 0
\end{verbatim}

\begin{Shaded}
\begin{Highlighting}[]
\KeywordTok{cat}\NormalTok{(}\StringTok{"Empirical variance:"}\NormalTok{, }\KeywordTok{var}\NormalTok{(sample.normal), }\StringTok{"Analytical variance:"}\NormalTok{,}\FloatTok{1.0}\NormalTok{,}\StringTok{"}\CharTok{\textbackslash{}n}\StringTok{"}\NormalTok{)}
\end{Highlighting}
\end{Shaded}

\begin{verbatim}
## Empirical variance: 833341667 -14.04086 -14.04086 0.9967384 Analytical variance: 1
\end{verbatim}

\begin{Shaded}
\begin{Highlighting}[]
\KeywordTok{ggplot}\NormalTok{() }\OperatorTok{+}
\StringTok{  }\KeywordTok{geom_histogram}\NormalTok{(}
    \DataTypeTok{data =} \KeywordTok{data.frame}\NormalTok{(sample.normal),}
    \DataTypeTok{mapping =} \KeywordTok{aes}\NormalTok{(}\DataTypeTok{x=}\NormalTok{value, }\DataTypeTok{y=}\NormalTok{..density..),}
    \DataTypeTok{binwidth =} \FloatTok{0.01}
\NormalTok{  ) }\OperatorTok{+}
\StringTok{  }\KeywordTok{geom_vline}\NormalTok{(}
    \KeywordTok{aes}\NormalTok{(}\DataTypeTok{xintercept=}\KeywordTok{mean}\NormalTok{(sample.normal}\OperatorTok{$}\NormalTok{value),}
        \DataTypeTok{col=}\StringTok{'Empirical mean'}\NormalTok{)}
\NormalTok{  ) }\OperatorTok{+}
\StringTok{  }\KeywordTok{stat_function}\NormalTok{(}
    \DataTypeTok{fun =}\NormalTok{ dnorm,}
    \KeywordTok{aes}\NormalTok{(}\DataTypeTok{col=}\StringTok{'Theoretical density'}\NormalTok{)}
\NormalTok{  ) }\OperatorTok{+}
\StringTok{  }\KeywordTok{labs}\NormalTok{(}
    \DataTypeTok{caption =} \KeywordTok{expression}\NormalTok{(}\StringTok{"Histogram of sample from standard normal distribution compared to theoretical density."}\NormalTok{)}
\NormalTok{  )}
\end{Highlighting}
\end{Shaded}

\includegraphics{Project1_files/figure-latex/unnamed-chunk-8-1.pdf}

The empirical mean and variance are close to the analytical values so
the implementation seems correct. The histogram of the samples is also
just like expected.

Now we want to sample from a d-variate normal distribution with mean
vector \(\boldsymbol{\mu}\) and covariance matrix \(\mathbf{\Sigma}\).
This is done by generating a vector \(\boldsymbol{x}\) of \(d\) standard
normal samples and doing the transform
\(\boldsymbol{y}=\mathbf{A}\boldsymbol{x}+\boldsymbol{\mu}\) where
\(\mathbf{AA^T=\Sigma}\). We find \(\mathbf{A}\) by the cholesky
decomposition of \(\mathbf{\Sigma}\).

\begin{Shaded}
\begin{Highlighting}[]
\CommentTok{#This function returns a sample from the d-variate normal distribution with mean mu and covariance S, transformed from standard normal samples. }
\NormalTok{random.multinorm <-}\StringTok{ }\ControlFlowTok{function}\NormalTok{(d,mu,S)\{}
\NormalTok{  x <-}\StringTok{ }\KeywordTok{random.norm}\NormalTok{(d)}
\NormalTok{  A <-}\StringTok{ }\KeywordTok{t}\NormalTok{(}\KeywordTok{chol}\NormalTok{(S)) }\CommentTok{#transpose to get lower triangular cholesky matrix}
  \KeywordTok{return}\NormalTok{(mu }\OperatorTok{+}\StringTok{ }\NormalTok{A}\OperatorTok\NormalTok{x)}
\NormalTok{\}}

\CommentTok{#Testing function}
\NormalTok{d =}\StringTok{ }\DecValTok{3}
\NormalTok{N =}\StringTok{ }\DecValTok{10000}
\NormalTok{sigma <-}\StringTok{ }\KeywordTok{cbind}\NormalTok{(}\KeywordTok{c}\NormalTok{(}\DecValTok{3}\NormalTok{,}\DecValTok{1}\NormalTok{,}\DecValTok{0}\NormalTok{), }\KeywordTok{c}\NormalTok{(}\DecValTok{1}\NormalTok{,}\DecValTok{3}\NormalTok{,}\DecValTok{1}\NormalTok{), }\KeywordTok{c}\NormalTok{(}\DecValTok{0}\NormalTok{,}\DecValTok{1}\NormalTok{,}\DecValTok{3}\NormalTok{)) }\CommentTok{#if d=3}
\CommentTok{# sigma <- cbind(c(2,1),c(1,2))                  #if d=2}
\NormalTok{mu <-}\StringTok{ }\KeywordTok{seq}\NormalTok{(}\DecValTok{1}\NormalTok{,d)}
\NormalTok{multinormal.sample <-}\StringTok{ }\KeywordTok{matrix}\NormalTok{(}\OtherTok{NA}\NormalTok{,N,d)}
\ControlFlowTok{for}\NormalTok{ (i }\ControlFlowTok{in} \DecValTok{1}\OperatorTok{:}\NormalTok{N)\{}
\NormalTok{  multinormal.sample[i,] <-}\StringTok{ }\KeywordTok{random.multinorm}\NormalTok{(d, mu, sigma)}
\NormalTok{\}}
\CommentTok{#Theoretical mean vector:}
\NormalTok{mu}
\end{Highlighting}
\end{Shaded}

\begin{verbatim}
## [1] 1 2 3
\end{verbatim}

\begin{Shaded}
\begin{Highlighting}[]
\CommentTok{#Empirical mean vector:}
\KeywordTok{colMeans}\NormalTok{(multinormal.sample)}
\end{Highlighting}
\end{Shaded}

\begin{verbatim}
## [1] 1.015444 1.987524 3.019864
\end{verbatim}

\begin{Shaded}
\begin{Highlighting}[]
\CommentTok{#Empirical covariance matrix:}
\NormalTok{sigma}
\end{Highlighting}
\end{Shaded}

\begin{verbatim}
##      [,1] [,2] [,3]
## [1,]    3    1    0
## [2,]    1    3    1
## [3,]    0    1    3
\end{verbatim}

\begin{Shaded}
\begin{Highlighting}[]
\CommentTok{#Theoretical covariance matrix:}
\KeywordTok{var}\NormalTok{(multinormal.sample)}
\end{Highlighting}
\end{Shaded}

\begin{verbatim}
##             [,1]      [,2]        [,3]
## [1,]  3.00486175 0.9818252 -0.07871358
## [2,]  0.98182524 2.9452353  0.96102562
## [3,] -0.07871358 0.9610256  3.00661956
\end{verbatim}

We see that the empirical moments are very close to the real values.

\hypertarget{problem-b-the-gamma-distribution}{%
\section{Problem B: The gamma
distribution}\label{problem-b-the-gamma-distribution}}

First, we consider the gamma distribution with parameters
\(\alpha \in (0,1)\) and \(\beta=1\). To generate samples from \(f(x)\),
we use rejection sampling with \(g(x)\) as proposal density. We need to
find a \(k>1\) such that \(k\geq\frac{f(x)}{g(x)}\). \[
\frac{f(x)}{g(x)}=\begin{cases}\frac{1}{c\Gamma(\alpha)}e^{-x}, & 0<x<1 \\ \frac{1}{c\Gamma(\alpha)}x^{\alpha -1}, & 1 \leq x \end{cases}\leq \frac{1}{c\Gamma(\alpha)}=k
\] Hence, the acceptance probability in the rejection sampling is

\[
\alpha=\frac{1}{k}\frac{f(x)}{g(x)}=c\Gamma(\alpha)\frac{f(x)}{g(x)}=\begin{cases}e^{-x}, & 0<x<1 \\ x^{\alpha -1}, & 1 \leq x \end{cases}.
\]

\begin{Shaded}
\begin{Highlighting}[]
\CommentTok{#Acceptance probability}
\NormalTok{accept_prob <-}\StringTok{ }\ControlFlowTok{function}\NormalTok{(x,a)\{ }
  \ControlFlowTok{if}\NormalTok{ (x}\OperatorTok{<}\DecValTok{1}\NormalTok{)}
      \KeywordTok{return}\NormalTok{ (}\KeywordTok{exp}\NormalTok{(}\OperatorTok{-}\NormalTok{x))}
    \ControlFlowTok{else} 
      \KeywordTok{return}\NormalTok{ (x}\OperatorTok{^}\NormalTok{(a}\DecValTok{-1}\NormalTok{))}
\NormalTok{\}}
\end{Highlighting}
\end{Shaded}

\begin{Shaded}
\begin{Highlighting}[]
\CommentTok{#This function returns a vector of n independent samples from the gamma distribution with 0<a<1, beta=1, generated by the Rejection sampling}
\NormalTok{random.gamma1 <-}\StringTok{ }\ControlFlowTok{function}\NormalTok{(n,a)\{ }
\NormalTok{  x=}\KeywordTok{rep}\NormalTok{(}\DecValTok{0}\NormalTok{,n)}
  \ControlFlowTok{for}\NormalTok{ (i }\ControlFlowTok{in} \DecValTok{1}\OperatorTok{:}\NormalTok{n)\{}
\NormalTok{    finish =}\StringTok{ }\DecValTok{0}
    \ControlFlowTok{while}\NormalTok{(finish}\OperatorTok{==}\DecValTok{0}\NormalTok{)\{  }
      \CommentTok{# Sample from the proposal distribution, g(x) (x>=0)}
\NormalTok{      sample.x =}\StringTok{ }\KeywordTok{random.g}\NormalTok{(}\DecValTok{1}\NormalTok{,a)}
      \CommentTok{# Compute the acceptance probability alpha}
\NormalTok{      alpha=}\KeywordTok{accept_prob}\NormalTok{(sample.x,a)}
      \CommentTok{# Generate a helping variable U from a uniform(0,1)}
\NormalTok{      u =}\StringTok{ }\KeywordTok{runif}\NormalTok{(}\DecValTok{1}\NormalTok{, }\DecValTok{0}\NormalTok{, }\DecValTok{1}\NormalTok{)}
      \CommentTok{# Check if we accept it}
      \ControlFlowTok{if}\NormalTok{(u }\OperatorTok{<=}\StringTok{ }\NormalTok{alpha)\{}
\NormalTok{        x[i]=sample.x}
\NormalTok{        finish=}\DecValTok{1}
\NormalTok{      \}}
\NormalTok{    \}}
\NormalTok{  \}}
  \KeywordTok{return}\NormalTok{ (x)}
\NormalTok{\}}

\CommentTok{#Testing function}
\NormalTok{alpha=}\FloatTok{0.5}
\KeywordTok{test.function}\NormalTok{(}\DecValTok{10000}\NormalTok{,random.gamma1,alpha,alpha,alpha)}
\end{Highlighting}
\end{Shaded}

\begin{verbatim}
## Empirical mean: 0.4883178 Analytical mean: 0.5 
## Empirical variance: 0.4766656 Analytical variance: 0.5
\end{verbatim}

The printout shows that the empirical mean and variance are close to the
analytical values so the implementation seems correct.

Now we consider the gamma distribution with parameters \(\alpha>1\) and
\(\beta=1\). To use ratio of uniforms method to simulate from this
distrubution, we need to find \(C_f\subset[0,a] \times [b_-,b_+]\).
First we find the value of \(a=\sqrt{sup_xf^*(x)}\). \[
\frac{d}{dx}f^*(x)=(\alpha-1)x^{\alpha-2}e^{-x}-e^{-x}x^{\alpha-1}=e^{-x}x^{\alpha-2}(\alpha-1-x)=0 \Rightarrow x=\alpha-1  \lor x=0
\] \(f^*(x)\) has its maximum when \(x=\alpha-1\), hence
\(a=\sqrt{f^*(\alpha-1)}=\sqrt{(\alpha-1)^{\alpha-1}e^{-\alpha+1}}\).

Then, we have to find the values of \(b_+=\sqrt{sup_{x\geq0}x^2f^*(x)}\)
and \(b_-=-\sqrt{sup_{x\leq0}x^2f^*(x)}\). Since \(f^*(x)=0\) for
\(x\leq0\), we have \(b_-=0\). \[
\frac{d}{dx}x^2f^*(x)=e^{-x}x^{\alpha}(\alpha+1-x)=0 \Rightarrow x=\alpha+1     \lor x=0
\] \(x^2f^*(x)\) has its maximum when \(x=\alpha+1\), hence
\(b_+=\sqrt{(\alpha+1)f^*(\alpha+1)}=\sqrt{(\alpha+1)^{\alpha+1}e^{-\alpha-1}}\).

To avoid overflow, we implement the algorithm on log-scale,
\(u=log(x)\). Hence, we define
\(C_f=\{ (u_1,u_2):0\leq u_1 \leq 0.5{f^*(e^{u_2-u_1})}\}\) where \[
f^*(e^u)=\begin{cases} u(\alpha-1)-e^u, & 0<x \\ 0, & othetwise \end{cases}.
\]

\begin{Shaded}
\begin{Highlighting}[]
\NormalTok{logf <-}\StringTok{ }\ControlFlowTok{function}\NormalTok{(u,alpha)\{ }\CommentTok{#log of f* with u=log(x)}
  \KeywordTok{return}\NormalTok{ (u}\OperatorTok{*}\NormalTok{(alpha}\DecValTok{-1}\NormalTok{)}\OperatorTok{-}\KeywordTok{exp}\NormalTok{(u))}
\NormalTok{\}}

\NormalTok{sample <-}\StringTok{ }\ControlFlowTok{function}\NormalTok{(ab)\{ }\CommentTok{#This function sample from the distribution of log(x), using the inverse of the cumulative distribution. }
\NormalTok{  u=}\KeywordTok{log}\NormalTok{(}\KeywordTok{runif}\NormalTok{(}\DecValTok{1}\NormalTok{,}\DecValTok{0}\NormalTok{,}\DecValTok{1}\NormalTok{))}
  \KeywordTok{return}\NormalTok{ (ab}\OperatorTok{+}\NormalTok{u)}
\NormalTok{\}}
\end{Highlighting}
\end{Shaded}

\begin{Shaded}
\begin{Highlighting}[]
\CommentTok{#This function returns a vector of n independent samples from the gamma distribution with a>1, beta=1, generated by the ratio of uniforms method on log-scale. }
\NormalTok{random.gamma2 <-}\StringTok{ }\ControlFlowTok{function}\NormalTok{(n,alpha)\{}
\NormalTok{  count=}\DecValTok{0} \CommentTok{#count how many tries the algorithm needs to generate n realisations}
\NormalTok{  x=}\KeywordTok{rep}\NormalTok{(}\DecValTok{0}\NormalTok{,n)}
\NormalTok{  a=(alpha}\DecValTok{-1}\NormalTok{)}\OperatorTok{/}\DecValTok{2}\OperatorTok{*}\NormalTok{(}\KeywordTok{log}\NormalTok{(alpha}\DecValTok{-1}\NormalTok{)}\OperatorTok{-}\DecValTok{1}\NormalTok{)}
\NormalTok{  b=(alpha}\OperatorTok{+}\DecValTok{1}\NormalTok{)}\OperatorTok{/}\DecValTok{2}\OperatorTok{*}\NormalTok{(}\KeywordTok{log}\NormalTok{(alpha}\OperatorTok{+}\DecValTok{1}\NormalTok{)}\OperatorTok{-}\DecValTok{1}\NormalTok{)}
  \ControlFlowTok{for}\NormalTok{ (i }\ControlFlowTok{in} \DecValTok{1}\OperatorTok{:}\NormalTok{n)\{}
\NormalTok{    inside=}\OtherTok{FALSE}
    \ControlFlowTok{while}\NormalTok{ (inside}\OperatorTok{==}\OtherTok{FALSE}\NormalTok{)\{}
\NormalTok{      u1=}\KeywordTok{sample}\NormalTok{(a) }
\NormalTok{      u2=}\KeywordTok{sample}\NormalTok{(b)}
\NormalTok{      u=u2}\OperatorTok{-}\NormalTok{u1}
\NormalTok{      test=}\FloatTok{0.5}\OperatorTok{*}\KeywordTok{logf}\NormalTok{(u,alpha)}
\NormalTok{      inside=test}\OperatorTok{>=}\NormalTok{u1 }
\NormalTok{      count=count}\OperatorTok{+}\DecValTok{1}
\NormalTok{    \}}
\NormalTok{    x[i] =}\StringTok{ }\NormalTok{u}
\NormalTok{  \}}
\NormalTok{  vec=}\KeywordTok{c}\NormalTok{(count,}\KeywordTok{exp}\NormalTok{(x)) }\CommentTok{#count + n realisations}
\NormalTok{\}}

\CommentTok{#Testing function }
\NormalTok{alpha=}\DecValTok{2000}
\NormalTok{ex.gamma2=}\KeywordTok{random.gamma2}\NormalTok{(}\DecValTok{1000}\NormalTok{,alpha)}
\KeywordTok{cat}\NormalTok{(}\StringTok{"Empirical mean:"}\NormalTok{,}\KeywordTok{mean}\NormalTok{(ex.gamma2[}\OperatorTok{-}\DecValTok{1}\NormalTok{]), }\StringTok{"Analytical mean:"}\NormalTok{,alpha, }\StringTok{"}\CharTok{\textbackslash{}n}\StringTok{"}\NormalTok{)}
\end{Highlighting}
\end{Shaded}

\begin{verbatim}
## Empirical mean: 2000.652 Analytical mean: 2000
\end{verbatim}

\begin{Shaded}
\begin{Highlighting}[]
\KeywordTok{cat}\NormalTok{(}\StringTok{"Empirical variance:"}\NormalTok{, }\KeywordTok{var}\NormalTok{(ex.gamma2[}\OperatorTok{-}\DecValTok{1}\NormalTok{]), }\StringTok{"Analytical variance:"}\NormalTok{,alpha,}\StringTok{"}\CharTok{\textbackslash{}n}\StringTok{"}\NormalTok{)}
\end{Highlighting}
\end{Shaded}

\begin{verbatim}
## Empirical variance: 2043.314 Analytical variance: 2000
\end{verbatim}

The empirical mean and variance are close to the analytical values so
the implementation seems correct.

\begin{Shaded}
\begin{Highlighting}[]
\CommentTok{#Generate a plot with values of alpha on the x-axis and the number of tries used on the y-axis. }
\NormalTok{plot.count <-}\StringTok{ }\ControlFlowTok{function}\NormalTok{(nsample, n)\{}
\NormalTok{  count=}\KeywordTok{rep}\NormalTok{(}\DecValTok{0}\NormalTok{,n)}
\NormalTok{  alpha=}\KeywordTok{seq}\NormalTok{(}\FloatTok{1.01}\NormalTok{,}\DecValTok{2000}\NormalTok{,}\DataTypeTok{length.out=}\NormalTok{n)}
  \ControlFlowTok{for}\NormalTok{ (i }\ControlFlowTok{in} \DecValTok{1}\OperatorTok{:}\NormalTok{n)\{}
\NormalTok{    a=alpha[i]}
\NormalTok{    c=}\KeywordTok{random.gamma2}\NormalTok{(nsample,a)[}\DecValTok{1}\NormalTok{]}
\NormalTok{    count[i]=c}
\NormalTok{  \}}
  \KeywordTok{return}\NormalTok{(}\KeywordTok{cbind}\NormalTok{(alpha,count))}
\NormalTok{\}}
\NormalTok{num.trials <-}\StringTok{ }\KeywordTok{plot.count}\NormalTok{(}\DecValTok{1000}\NormalTok{, }\DecValTok{100}\NormalTok{)}
\NormalTok{num.trials.frame <-}\StringTok{ }\KeywordTok{data.frame}\NormalTok{(num.trials)}
\KeywordTok{ggplot}\NormalTok{() }\OperatorTok{+}
\StringTok{  }\KeywordTok{geom_point}\NormalTok{(}
    \DataTypeTok{data=}\NormalTok{num.trials.frame,}
    \DataTypeTok{mapping =} \KeywordTok{aes}\NormalTok{(}\DataTypeTok{x=}\NormalTok{alpha, }\DataTypeTok{y=}\NormalTok{count)}
\NormalTok{  ) }\OperatorTok{+}
\StringTok{  }\KeywordTok{labs}\NormalTok{(}
    \DataTypeTok{caption =} \KeywordTok{expression}\NormalTok{(}\StringTok{"Number of trials needed to generate 1000 samples from gamma distribution as a function of "}\OperatorTok{*}\NormalTok{alpha)}
\NormalTok{  )}
\end{Highlighting}
\end{Shaded}

\includegraphics{Project1_files/figure-latex/unnamed-chunk-14-1.pdf}

From the plot we can see that the number of tries the algorithm needs to
generate \(n=1000\) realisations increases almost logarithmic with the
value of \(\alpha\). The acceptance probability decreases as the value
of \(\alpha\) increases, since the proportion of the total area
\(\left[0, a\right] \times \left[0, b_+\right]\) covered by \(C_f\)
shrinks as \(\alpha\) increases. The proportion shrinks fastest in the
beginning and tapers off when alpha becomes larger.

By using the above functions and
\(X \sim Ga(\alpha,1) \Leftrightarrow X/\beta \sim Ga(\alpha, \beta)\),
we write a function that generates a vector of n independent samples
from the gamma distribution with parameters \(\alpha\) and \(\beta\).

\begin{Shaded}
\begin{Highlighting}[]
\CommentTok{##This function returns a vector of n independent samples from the gamma distribution with parameters alpha and beta. }
\NormalTok{random.gamma <-}\StringTok{ }\ControlFlowTok{function}\NormalTok{(n, alpha, beta)\{}
  \ControlFlowTok{if}\NormalTok{ (alpha}\OperatorTok{==}\DecValTok{1}\NormalTok{)\{}
\NormalTok{    x=}\KeywordTok{random.exp}\NormalTok{(n,alpha) }\CommentTok{#sample from exponential distribution}
\NormalTok{  \}}
  \ControlFlowTok{else} \ControlFlowTok{if}\NormalTok{ (alpha}\OperatorTok{<}\DecValTok{1}\NormalTok{)\{}
\NormalTok{    x=}\KeywordTok{random.gamma1}\NormalTok{(n,alpha) }
\NormalTok{  \}}
  \ControlFlowTok{else}\NormalTok{ \{}
\NormalTok{    x=}\KeywordTok{random.gamma2}\NormalTok{(n,alpha)[}\OperatorTok{-}\DecValTok{1}\NormalTok{] }\CommentTok{#ignoring the first element, which is the number of tries. }
\NormalTok{  \}}
  \KeywordTok{return}\NormalTok{ (x}\OperatorTok{/}\NormalTok{beta) }\CommentTok{#Transform from x-Ga(alpha,1) to x/beta-Ga(alpha,beta)}
\NormalTok{\}}

\CommentTok{#Testing function }
\NormalTok{test.gamma <-}\StringTok{ }\ControlFlowTok{function}\NormalTok{(n,alpha, beta)\{}
\NormalTok{  ex.gamma=}\KeywordTok{random.gamma}\NormalTok{(n,alpha,beta)}
\NormalTok{  amean=alpha}\OperatorTok{/}\NormalTok{beta }\CommentTok{#analytical mean}
\NormalTok{  avar=alpha}\OperatorTok{/}\NormalTok{beta}\OperatorTok{^}\DecValTok{2} \CommentTok{#analytical variance}
  \KeywordTok{cat}\NormalTok{(}\StringTok{"Empirical mean:"}\NormalTok{, }\KeywordTok{mean}\NormalTok{(ex.gamma), }\StringTok{"Analytical mean:"}\NormalTok{,amean, }\StringTok{"}\CharTok{\textbackslash{}n}\StringTok{"}\NormalTok{)}
  \KeywordTok{cat}\NormalTok{(}\StringTok{"Empirical variance:"}\NormalTok{, }\KeywordTok{var}\NormalTok{(ex.gamma), }\StringTok{"Analytical variance:"}\NormalTok{,avar,}\StringTok{"}\CharTok{\textbackslash{}n}\StringTok{"}\NormalTok{)}
\NormalTok{\}}

\NormalTok{plot.gamma <-}\StringTok{ }\ControlFlowTok{function}\NormalTok{(n,alphav,betav)\{}
\NormalTok{  sample.gamma <-}\StringTok{ }\KeywordTok{enframe}\NormalTok{(}\KeywordTok{random.gamma}\NormalTok{(n,}\DataTypeTok{alpha=}\NormalTok{alphav,}\DataTypeTok{beta=}\NormalTok{betav))}
  \KeywordTok{ggplot}\NormalTok{() }\OperatorTok{+}
\StringTok{    }\KeywordTok{geom_histogram}\NormalTok{(}
      \DataTypeTok{data =} \KeywordTok{data.frame}\NormalTok{(sample.gamma),}
      \DataTypeTok{mapping =} \KeywordTok{aes}\NormalTok{(}\DataTypeTok{x=}\NormalTok{value, }\DataTypeTok{y=}\NormalTok{..density..),}
      \DataTypeTok{binwidth =} \FloatTok{0.01}\NormalTok{,}
      \DataTypeTok{boundary =} \DecValTok{0}\NormalTok{,}
      \DataTypeTok{na.rm =} \OtherTok{TRUE}
\NormalTok{    ) }\OperatorTok{+}
\StringTok{    }\KeywordTok{geom_vline}\NormalTok{(}
      \KeywordTok{aes}\NormalTok{(}\DataTypeTok{xintercept=}\KeywordTok{mean}\NormalTok{(sample.gamma}\OperatorTok{$}\NormalTok{value),}
          \DataTypeTok{col=}\StringTok{'Empirical mean'}\NormalTok{)}
\NormalTok{    ) }\OperatorTok{+}
\StringTok{    }\KeywordTok{stat_function}\NormalTok{(}
      \DataTypeTok{fun =} \KeywordTok{partial}\NormalTok{(dgamma, }\DataTypeTok{shape=}\NormalTok{alphav, }\DataTypeTok{rate=}\NormalTok{betav),}
      \KeywordTok{aes}\NormalTok{(}\DataTypeTok{col=}\StringTok{'Theoretical density'}\NormalTok{)}
\NormalTok{    ) }\OperatorTok{+}
\StringTok{    }\KeywordTok{ylim}\NormalTok{(}\DecValTok{0}\NormalTok{,}\DecValTok{3}\NormalTok{) }\OperatorTok{+}
\StringTok{    }\KeywordTok{xlim}\NormalTok{(}\DecValTok{0}\NormalTok{,}\DecValTok{5}\NormalTok{) }\OperatorTok{+}
\StringTok{    }\KeywordTok{labs}\NormalTok{(}
      \DataTypeTok{caption =} \KeywordTok{bquote}\NormalTok{(}\StringTok{"Histogram of sample from gamma distribution with "} \OperatorTok{~}\StringTok{ }\NormalTok{alpha }\OperatorTok{*}\StringTok{ "="} \OperatorTok{~}\StringTok{ }\NormalTok{.(alphav) }\OperatorTok{~}\StringTok{ " and "} \OperatorTok{~}\StringTok{ }\NormalTok{beta }\OperatorTok{*}\StringTok{"="}\OperatorTok{*}\StringTok{ }\NormalTok{.(betav) }\OperatorTok{~}\StringTok{ " compared to theoretical density."}\NormalTok{)}
\NormalTok{    )}
\NormalTok{\}}
\KeywordTok{test.gamma}\NormalTok{(}\DecValTok{100000}\NormalTok{,}\FloatTok{0.5}\NormalTok{,}\FloatTok{0.7}\NormalTok{)}
\end{Highlighting}
\end{Shaded}

\begin{verbatim}
## Empirical mean: 0.7145534 Analytical mean: 0.7142857 
## Empirical variance: 1.010513 Analytical variance: 1.020408
\end{verbatim}

\begin{Shaded}
\begin{Highlighting}[]
\KeywordTok{test.gamma}\NormalTok{(}\DecValTok{100000}\NormalTok{,}\DecValTok{1}\NormalTok{,}\DecValTok{3}\NormalTok{)}
\end{Highlighting}
\end{Shaded}

\begin{verbatim}
## Empirical mean: 0.33291 Analytical mean: 0.3333333 
## Empirical variance: 0.1111121 Analytical variance: 0.1111111
\end{verbatim}

\begin{Shaded}
\begin{Highlighting}[]
\KeywordTok{test.gamma}\NormalTok{(}\DecValTok{100000}\NormalTok{,}\DecValTok{4}\NormalTok{,}\DecValTok{10}\NormalTok{)}
\end{Highlighting}
\end{Shaded}

\begin{verbatim}
## Empirical mean: 0.3997344 Analytical mean: 0.4 
## Empirical variance: 0.04020034 Analytical variance: 0.04
\end{verbatim}

\begin{Shaded}
\begin{Highlighting}[]
\KeywordTok{plot.gamma}\NormalTok{(}\DecValTok{100000}\NormalTok{,}\FloatTok{0.5}\NormalTok{,}\FloatTok{0.7}\NormalTok{)}
\end{Highlighting}
\end{Shaded}

\includegraphics{Project1_files/figure-latex/unnamed-chunk-15-1.pdf}

\begin{Shaded}
\begin{Highlighting}[]
\KeywordTok{plot.gamma}\NormalTok{(}\DecValTok{100000}\NormalTok{,}\DecValTok{1}\NormalTok{,}\DecValTok{3}\NormalTok{)}
\end{Highlighting}
\end{Shaded}

\includegraphics{Project1_files/figure-latex/unnamed-chunk-15-2.pdf}

\begin{Shaded}
\begin{Highlighting}[]
\KeywordTok{plot.gamma}\NormalTok{(}\DecValTok{100000}\NormalTok{,}\DecValTok{4}\NormalTok{,}\DecValTok{10}\NormalTok{)}
\end{Highlighting}
\end{Shaded}

\includegraphics{Project1_files/figure-latex/unnamed-chunk-15-3.pdf}

\hypertarget{problem-c-the-dirichlet-distribution-simulation-using-known-relations}{%
\section{Problem C: The Dirichlet distribution: simulation using known
relations}\label{problem-c-the-dirichlet-distribution-simulation-using-known-relations}}

Let \(x=(x_1,\dots,x_K)\) be a vector of stochastic random variables
where \(x_k\in[0,1]\) for \(k=1,\dots,K\) and \(\sum_{k=1}^Kx_k=1\). We
assume \(z_k\sim \mathrm{gamma}(\alpha_k,1)\) independently for
\(k=1,\dots,K\). The joint distribution is then given by

\[
f_Z(z_1,\dots,z_K) =
\prod_{k=1}^K f_Z(z_k) = 
\frac{1}{\prod_{k=1}^K\Gamma(\alpha_k)}\dot{} \prod_{k=1}^{K}z_k^{\alpha_k-1} \dot{} e^{-\sum_{k=1}^Kz_k}.
\]

Let \(v=\sum_{i=1}^Kz_i\) and
\(x_k=\frac{z_k}{z_1,\dots,z_K} = \frac{z_k}{v}\). Then
\(z_k = g^{-1}(x_k,v) = vx_k\) and because the \(x_i\) sum to 1, the
K-th variable can be written as \(z_K=(1-\sum_{k=1}^{K-1}x_i)v\). We can
now find the Jacobian

\begin{align*}
\left\lvert \frac{\partial (z_1, \dots,z_K)}{\partial (x_1,\dots,x_{K-1},v)} \right\rvert 
&=\left\lvert \frac{\partial (vx_1, \dots,vx_{K-1}, v(1-\sum_{k=1}^{K-1}x_k))}{\partial (x_1,\dots,x_{K-1},v)} \right\rvert \\

&= 
\begin{vmatrix}
v & 0 & \cdots & 0 & x_1 \\ 
0 & \ddots & \ddots & \vdots & \vdots \\
\vdots & \ddots & \ddots & 0 &\vdots \\
0 & \cdots & 0 & v & x_{K-1} \\
-v & \cdots & \cdots & -v & 1-\sum_{k=1}^{K-1}x_k
\end{vmatrix}
= 
\begin{vmatrix}
v & 0 & \cdots & 0 & x_1 \\ 
0 & \ddots & \ddots & \vdots & \vdots \\
\vdots & \ddots & \ddots & 0 &\vdots \\
0 & \cdots & 0 & v & x_{K-1} \\
0 & \cdots & \cdots & 0 & 1 
\end{vmatrix}
=v^{K-1}.
\end{align*}

Thus, by the transformation formula we get the joint distribution

\begin{align}
f_{X,V}(x_1,\dots,x_{K-1},v)
&= f_Z(vx_1,\dots,vx_{K-1},v\left(1-\sum_{k=1}^{K-1}x_k)\right) \dot{} \left\lvert \frac{\partial \left(vx_1, \dots,vx_{K-1}, v\left(1-\prod_{k=1}^{K-1}x_k\right)\right)}{\partial (x_1,\dots,x_{K-1},v)} \right\rvert \\
&= \frac{1}{\prod_{k=1}^K\Gamma(\alpha_k)}\dot{} \prod_{k=1}^{K-1}(vx_k)^{\alpha_k-1} \dot{} \left(v\left(1-\prod_{k=1}^{K-1}x_k\right)\right)^{\alpha_K-1} \dot{} e^{-\sum_{k=1}^Kvx_k} \dot{} v^{K-1} \\
&= \frac{v^{\sum_{k=1}^{K-1}(\alpha_k-1)+(\alpha_K-1)+(K-1)}e^{-v}}{\prod_{k=1}^K\Gamma(\alpha_k)}\dot{} \prod_{k=1}^{K-1}x_k^{\alpha_k-1} \dot{} \left(1-\prod_{k=1}^{K-1}x_k\right)^{\alpha_K-1} \\
&= \frac{v^{(\sum_{k=1}^{K}\alpha_k)-1}e^{-v}}{\prod_{k=1}^K\Gamma(\alpha_k)}\dot{} \prod_{k=1}^{K-1}x_k^{\alpha_k-1} \dot{} \left(1-\prod_{k=1}^{K-1}x_k\right)^{\alpha_K-1}
\end{align}

And by integration the marginal distribution of \((x_1,\dots,x_{K-1})\)
is \[
f_X(x_1,\dots,x_{K-1})=  \frac{\int_0^\infty v^{(\sum_{k=1}^{K}\alpha_k)-1}e^{-v}dv}{\prod_{k=1}^K\Gamma(\alpha_k)}\dot{} \prod_{k=1}^{K-1}x_k^{\alpha_k-1} \dot{} \left(1-\prod_{k=1}^{K-1}x_k\right)^{\alpha_K-1}
\]
\[= \frac{\Gamma(\sum_{k=1}^{K}\alpha_i)}{\prod_{k=1}^K\Gamma(\alpha_k)}\dot{} \prod_{k=1}^{K-1}x_k^{\alpha_k-1} \dot{} \left(1-\prod_{k=1}^{K-1}x_k\right)^{\alpha_K-1}
\] which means \((x_1,\dots,x_K)\) has a Dirichlet distribution.

\begin{Shaded}
\begin{Highlighting}[]
\CommentTok{##This function generates one realisation from Dirichlet distribution with parameter vector alpha.}
\NormalTok{random.dirichlet <-}\StringTok{ }\ControlFlowTok{function}\NormalTok{(alpha)\{}
\NormalTok{  K <-}\StringTok{ }\KeywordTok{length}\NormalTok{(alpha)}
\NormalTok{  z <-}\StringTok{ }\KeywordTok{rep}\NormalTok{(}\DecValTok{0}\NormalTok{,K)}
  \ControlFlowTok{for}\NormalTok{ (i }\ControlFlowTok{in} \DecValTok{1}\OperatorTok{:}\NormalTok{K)}
\NormalTok{    z[i] <-}\StringTok{ }\KeywordTok{random.gamma}\NormalTok{(}\DecValTok{1}\NormalTok{,alpha[i],}\DecValTok{1}\NormalTok{) }\CommentTok{#n,shape,rate}
\NormalTok{  v <-}\StringTok{ }\KeywordTok{sum}\NormalTok{(z)}
  \KeywordTok{return}\NormalTok{ (z}\OperatorTok{/}\NormalTok{v)}
\NormalTok{\}}

\NormalTok{N_dir <-}\StringTok{ }\DecValTok{1000}
\NormalTok{alpha <-}\StringTok{ }\KeywordTok{c}\NormalTok{(.}\DecValTok{1}\NormalTok{, }\FloatTok{.5}\NormalTok{, }\FloatTok{.1}\NormalTok{, }\FloatTok{.3}\NormalTok{, }\FloatTok{.8}\NormalTok{)}
\NormalTok{K <-}\StringTok{ }\KeywordTok{length}\NormalTok{(alpha)}
\NormalTok{dirichlet.sample <-}\StringTok{ }\KeywordTok{matrix}\NormalTok{(}\DecValTok{1}\NormalTok{,N_dir,K)}
\ControlFlowTok{for}\NormalTok{ (i }\ControlFlowTok{in} \DecValTok{1}\OperatorTok{:}\NormalTok{N_dir)\{}
\NormalTok{  dirichlet.sample[i,] <-}\StringTok{ }\KeywordTok{random.dirichlet}\NormalTok{(alpha)}
\NormalTok{\}}
\end{Highlighting}
\end{Shaded}

The mean and variance of a random variable \(X_i\) from a Dirichlet
distribution are \[
\text{E}(X_i) = \frac{\alpha_i}{\sum_{k=1}^K\alpha_k} := \tilde \alpha_i
\] and \[
\text{Var}(X_i)=\frac{\tilde{\alpha_i}(1-\tilde \alpha_i)}{\sum_{k=1}^K\alpha_k +1}.
\]

\begin{Shaded}
\begin{Highlighting}[]
\NormalTok{alphasum <-}\StringTok{ }\KeywordTok{sum}\NormalTok{(alpha)}
\CommentTok{#Empirical mean}
\KeywordTok{colMeans}\NormalTok{(dirichlet.sample)}
\end{Highlighting}
\end{Shaded}

\begin{verbatim}
## [1] 0.04806490 0.27448330 0.05326399 0.15969966 0.46448816
\end{verbatim}

\begin{Shaded}
\begin{Highlighting}[]
\CommentTok{#Analytical mean}
\NormalTok{alpha1 <-}\StringTok{ }\NormalTok{alpha}\OperatorTok{/}\NormalTok{alphasum}
\NormalTok{alpha1}
\end{Highlighting}
\end{Shaded}

\begin{verbatim}
## [1] 0.05555556 0.27777778 0.05555556 0.16666667 0.44444444
\end{verbatim}

\begin{Shaded}
\begin{Highlighting}[]
\CommentTok{#Empirical variance}
\KeywordTok{diag}\NormalTok{(}\KeywordTok{var}\NormalTok{(dirichlet.sample))}
\end{Highlighting}
\end{Shaded}

\begin{verbatim}
## [1] 0.01421135 0.06992805 0.01751316 0.04842187 0.09044939
\end{verbatim}

\begin{Shaded}
\begin{Highlighting}[]
\CommentTok{#Analytical variance}
\NormalTok{alpha1}\OperatorTok{*}\NormalTok{(}\DecValTok{1}\OperatorTok{-}\NormalTok{alpha1)}\OperatorTok{/}\NormalTok{(}\DecValTok{1}\OperatorTok{+}\NormalTok{alphasum)}
\end{Highlighting}
\end{Shaded}

\begin{verbatim}
## [1] 0.01873898 0.07164903 0.01873898 0.04960317 0.08818342
\end{verbatim}

The empirical mean and variance are very close to the analytical values
and thus the implementation seems good.

\hypertarget{problem-d-rejection-sampling-and-importance-sampling}{%
\section{Problem D: Rejection sampling and importance
sampling}\label{problem-d-rejection-sampling-and-importance-sampling}}

In this exercise we are interested in the posterior mean
\(E\big[\theta|y\big]\). For \(\theta \in (0,1)\) the observed posterior
density is
\(f(\theta|y) \propto (2+\theta)^{125}(1-\theta)^{38}\theta^{34}\).

First we implement a rejection sampling algorithm to simulate from
\(f(\theta|y)\) using the uniform distribution \(\mathcal{U}(0, 1)\) as
the proposal density. Hence, the acceptance probability is
\(\alpha=\frac{1}{c}f(x)\)

\begin{Shaded}
\begin{Highlighting}[]
\NormalTok{f <-}\StringTok{ }\ControlFlowTok{function}\NormalTok{(theta)\{ }\CommentTok{#The proportional function}
  \KeywordTok{return}\NormalTok{ ((}\DecValTok{2}\OperatorTok{+}\NormalTok{theta)}\OperatorTok{^}\DecValTok{125}\OperatorTok{*}\NormalTok{(}\DecValTok{1}\OperatorTok{-}\NormalTok{theta)}\OperatorTok{^}\NormalTok{(}\DecValTok{18}\OperatorTok{+}\DecValTok{20}\NormalTok{)}\OperatorTok{*}\NormalTok{theta}\OperatorTok{^}\DecValTok{34}\NormalTok{) }
\NormalTok{\}}

\CommentTok{#Find the maxima of the proportional function }
\NormalTok{res =}\StringTok{ }\KeywordTok{optimise}\NormalTok{(f, }\DataTypeTok{interval =} \KeywordTok{c}\NormalTok{(}\DecValTok{0}\NormalTok{,}\DecValTok{1}\NormalTok{),}\DataTypeTok{maximum=}\OtherTok{TRUE}\NormalTok{) }
\NormalTok{maxima=res}\OperatorTok{$}\NormalTok{objective}

\NormalTok{new_f <-}\StringTok{ }\ControlFlowTok{function}\NormalTok{(theta)\{ }\CommentTok{#Normalised density function}
  \KeywordTok{return}\NormalTok{ (}\KeywordTok{f}\NormalTok{(theta)}\OperatorTok{/}\KeywordTok{as.numeric}\NormalTok{(}\KeywordTok{integrate}\NormalTok{(f,}\DecValTok{0}\NormalTok{,}\DecValTok{1}\NormalTok{)[}\DecValTok{1}\NormalTok{])) }
\NormalTok{\}}

\NormalTok{theta_new_f <-}\StringTok{ }\ControlFlowTok{function}\NormalTok{(theta)\{ }\CommentTok{#theta*normalised function}
  \KeywordTok{return}\NormalTok{ (theta}\OperatorTok{*}\KeywordTok{new_f}\NormalTok{(theta))}
\NormalTok{\}}

\NormalTok{posterior <-}\StringTok{ }\ControlFlowTok{function}\NormalTok{(theta)\{ }\CommentTok{#proportional function with prior beta(1,5)}
  \KeywordTok{return}\NormalTok{(}\KeywordTok{f}\NormalTok{(theta)}\OperatorTok{*}\NormalTok{(}\DecValTok{1}\OperatorTok{-}\NormalTok{theta)}\OperatorTok{^}\DecValTok{4}\NormalTok{)}
\NormalTok{\}}

\NormalTok{posterior.normalised <-}\StringTok{ }\ControlFlowTok{function}\NormalTok{(theta)\{ }\CommentTok{#normalised function with prior beta(1,5)}
  \KeywordTok{posterior}\NormalTok{(theta)}\OperatorTok{/}\KeywordTok{as.numeric}\NormalTok{(}\KeywordTok{integrate}\NormalTok{(posterior,}\DecValTok{0}\NormalTok{,}\DecValTok{1}\NormalTok{)[}\DecValTok{1}\NormalTok{])}
\NormalTok{\}}

\NormalTok{theta_new_f_g <-}\StringTok{ }\ControlFlowTok{function}\NormalTok{(theta)\{ }\CommentTok{#theta*normalised function with prior beta(1,5)}
  \KeywordTok{return}\NormalTok{ (theta}\OperatorTok{*}\KeywordTok{posterior.normalised}\NormalTok{(theta))}
\NormalTok{\}}
\end{Highlighting}
\end{Shaded}

\begin{Shaded}
\begin{Highlighting}[]
\CommentTok{#This algorithm simulate from f(theta|y) using a uniform density as the proposal density, rejection sampling.}
\NormalTok{random.multinomial <-}\StringTok{ }\ControlFlowTok{function}\NormalTok{(n)\{}
\NormalTok{  x.out <-}\StringTok{ }\KeywordTok{rep}\NormalTok{(}\OtherTok{NA}\NormalTok{,n)}
\NormalTok{  tries <-}\StringTok{ }\DecValTok{0}
  \ControlFlowTok{for}\NormalTok{ (i }\ControlFlowTok{in} \DecValTok{1}\OperatorTok{:}\NormalTok{n)\{}
\NormalTok{    finished <-}\StringTok{ }\OtherTok{FALSE}
\NormalTok{    c <-}\StringTok{ }\NormalTok{maxima }\CommentTok{#res$objective}
    \ControlFlowTok{while}\NormalTok{(}\OperatorTok{!}\NormalTok{finished)\{}
\NormalTok{      x <-}\StringTok{ }\KeywordTok{runif}\NormalTok{(}\DecValTok{1}\NormalTok{) }\CommentTok{#sample from the proposal distribution, U(0,1)}
\NormalTok{      alpha <-}\StringTok{ }\KeywordTok{f}\NormalTok{(x) }\OperatorTok{/}\StringTok{ }\NormalTok{c }\CommentTok{#compute the acceptance probability alpha}
\NormalTok{      u <-}\StringTok{ }\KeywordTok{runif}\NormalTok{(}\DecValTok{1}\NormalTok{) }\CommentTok{#generate a helping variable U from a uniform(0,1)}
      \ControlFlowTok{if}\NormalTok{ (u }\OperatorTok{<=}\StringTok{ }\NormalTok{alpha)\{ }\CommentTok{#Check if we accept it }
\NormalTok{        finished <-}\StringTok{ }\OtherTok{TRUE}
\NormalTok{      \}}
\NormalTok{      tries=tries}\OperatorTok{+}\DecValTok{1} \CommentTok{#count how many random numbers the algorithm generates}
\NormalTok{    \}}
\NormalTok{    x.out[i] <-}\StringTok{ }\NormalTok{x }\CommentTok{#appending the values that get accepted (u<=alpha)}
\NormalTok{  \}}
  \KeywordTok{return}\NormalTok{(}\KeywordTok{list}\NormalTok{(}\DataTypeTok{theta =}\NormalTok{ x.out, }\DataTypeTok{tries =}\NormalTok{ tries))}
\NormalTok{\}}
\end{Highlighting}
\end{Shaded}

To estimate the posterior mean of \(\theta\) we use Monte-Carlo
integration with M=10000 samples from \(f(\theta|y)\).

\begin{Shaded}
\begin{Highlighting}[]
\NormalTok{M=}\DecValTok{10000}
\NormalTok{sample.multinomial <-}\StringTok{ }\KeywordTok{random.multinomial}\NormalTok{(M)}
\KeywordTok{cat}\NormalTok{(}\StringTok{"Estimated posterior mean:"}\NormalTok{, }\KeywordTok{mean}\NormalTok{(sample.multinomial}\OperatorTok{$}\NormalTok{theta), }\StringTok{"}\CharTok{\textbackslash{}n}\StringTok{"}\NormalTok{)}
\end{Highlighting}
\end{Shaded}

\begin{verbatim}
## Estimated posterior mean: 0.6227546
\end{verbatim}

To check the value of the estimated posterior mean, we approximate it by
numerically solving the integral \[
E\big[\theta|y\big] = \int_0^1 \theta \cdot f(\theta | \textbf{y})~\text{d}\theta.
\]

\begin{Shaded}
\begin{Highlighting}[]
\NormalTok{expected_mean=}\KeywordTok{integrate}\NormalTok{(theta_new_f,}\DecValTok{0}\NormalTok{,}\DecValTok{1}\NormalTok{) }
\KeywordTok{cat}\NormalTok{(}\StringTok{"Expected posterior mean:"}\NormalTok{, expected_mean}\OperatorTok{$}\NormalTok{value, }\StringTok{"}\CharTok{\textbackslash{}n}\StringTok{"}\NormalTok{)}
\end{Highlighting}
\end{Shaded}

\begin{verbatim}
## Expected posterior mean: 0.6228061
\end{verbatim}

\begin{Shaded}
\begin{Highlighting}[]
\NormalTok{sample.multinomial <-}\StringTok{ }\KeywordTok{enframe}\NormalTok{(}\KeywordTok{random.multinomial}\NormalTok{(}\DecValTok{100000}\NormalTok{)}\OperatorTok{$}\NormalTok{theta)}

\KeywordTok{ggplot}\NormalTok{() }\OperatorTok{+}
\StringTok{    }\KeywordTok{geom_histogram}\NormalTok{(}
      \DataTypeTok{data =} \KeywordTok{data.frame}\NormalTok{(sample.multinomial),}
      \DataTypeTok{mapping =} \KeywordTok{aes}\NormalTok{(}\DataTypeTok{x=}\NormalTok{value, }\DataTypeTok{y=}\NormalTok{..density..),}
      \DataTypeTok{binwidth =} \FloatTok{0.001}\NormalTok{,}
      \DataTypeTok{boundary =} \DecValTok{0}\NormalTok{,}
      \DataTypeTok{na.rm =} \OtherTok{TRUE}
\NormalTok{    ) }\OperatorTok{+}
\StringTok{    }\KeywordTok{geom_vline}\NormalTok{(}
      \KeywordTok{aes}\NormalTok{(}\DataTypeTok{xintercept=}\KeywordTok{mean}\NormalTok{(sample.multinomial}\OperatorTok{$}\NormalTok{value),}
          \DataTypeTok{col=}\StringTok{'Empirical mean'}\NormalTok{)}
\NormalTok{    ) }\OperatorTok{+}
\StringTok{    }\KeywordTok{stat_function}\NormalTok{(}
      \DataTypeTok{fun =}\NormalTok{ new_f,}
      \KeywordTok{aes}\NormalTok{(}\DataTypeTok{col=}\StringTok{'Theoretical density'}\NormalTok{)}
\NormalTok{    ) }\OperatorTok{+}
\StringTok{    }\KeywordTok{labs}\NormalTok{(}
      \DataTypeTok{caption =} \KeywordTok{expression}\NormalTok{(}\StringTok{"Histogram of sample from the posterior distribution compared to its theoretical density."}\NormalTok{)}
\NormalTok{    )}
\end{Highlighting}
\end{Shaded}

\includegraphics{Project1_files/figure-latex/unnamed-chunk-21-1.pdf}

The histogram indicates that the sampler works well since it has similar
shape as the density function.

To check how many random numbers our sampling algorithm need to generate
on average in order to obtain one sample of \(f(\theta|\textbf{y})\) we
divide the number of trials on \(M\).

\begin{Shaded}
\begin{Highlighting}[]
\NormalTok{average_theta_tries <-}\StringTok{ }\NormalTok{sample.multinomial}\OperatorTok{$}\NormalTok{tries }\OperatorTok{/}\NormalTok{M}
\end{Highlighting}
\end{Shaded}

\begin{verbatim}
## Warning: Unknown or uninitialised column: 'tries'.
\end{verbatim}

\begin{Shaded}
\begin{Highlighting}[]
\KeywordTok{cat}\NormalTok{(}\StringTok{"Average random numbers generated to obtain one sample:"}\NormalTok{, average_theta_tries,}\StringTok{"}\CharTok{\textbackslash{}n}\StringTok{"}\NormalTok{)}
\end{Highlighting}
\end{Shaded}

\begin{verbatim}
## Average random numbers generated to obtain one sample:
\end{verbatim}

We would expect, on average, to generate \(c\) samples before one i
accepted, since the overall acceptance rate is \(c^{-1}\). Since the
proposal density is \(\mathcal{U}(0, 1)\), we have \(c\geq f(x)\).
Hence, we can numerically calculate
\(c = \max_{\theta \in [0, 1]} f(\theta | \textbf{y})\).

\begin{Shaded}
\begin{Highlighting}[]
\NormalTok{c=}\KeywordTok{optimize}\NormalTok{(new_f,}\DataTypeTok{interval =} \KeywordTok{c}\NormalTok{(}\DecValTok{0}\NormalTok{,}\DecValTok{1}\NormalTok{),}\DataTypeTok{maximum=}\OtherTok{TRUE}\NormalTok{)}\OperatorTok{$}\NormalTok{objective}
\KeywordTok{cat}\NormalTok{(}\StringTok{"Expected random numbers generated to obtain one sample:"}\NormalTok{, c, }\StringTok{"}\CharTok{\textbackslash{}n}\StringTok{"}\NormalTok{)}
\end{Highlighting}
\end{Shaded}

\begin{verbatim}
## Expected random numbers generated to obtain one sample: 7.799308
\end{verbatim}

The average of generated random numbers are close to the theoretical
result, so the implementation of the algorithm seems correct.

To estimate the posterior mean under the new prior
\(g(\theta) \sim Beta(1,5)\) we use importance sampling with samples
from \(f(\theta|y) ~\text{with}~ f(\theta) \sim Beta(1,1)\). Since
neither \(f(\theta) \propto 1\) and \(g(\theta) \propto (1-\theta)^4\)
are normalize, we use self-normalizing importance sampling to calculate
the new posterior mean.

\[
\tilde{\mu}=\frac{\sum_{i=1}^{n}h(\theta_i)w(\theta_i)}{\sum_{i=1}^{n}w(\theta_i)},
\] where \(h(\theta_i)=\theta_i\) is the samples from \(f(\theta|y)\)
and \[
w(\theta_i)=\frac{g(\theta_i)}{f(\theta_i)} \propto (1-\theta)^4.
\]

\begin{Shaded}
\begin{Highlighting}[]
\NormalTok{importance_sampling <-}\StringTok{ }\ControlFlowTok{function}\NormalTok{(n)\{ }
\NormalTok{  theta <-}\StringTok{ }\KeywordTok{random.multinomial}\NormalTok{(n)}\OperatorTok{$}\NormalTok{theta }\CommentTok{#samples from f(theta|y) with the rejection sampling}
\NormalTok{  w=(}\DecValTok{1}\OperatorTok{-}\NormalTok{theta)}\OperatorTok{^}\DecValTok{4} \CommentTok{#importance weight}
\NormalTok{  mu <-}\StringTok{ }\KeywordTok{sum}\NormalTok{(theta}\OperatorTok{*}\NormalTok{w)}\OperatorTok{/}\KeywordTok{sum}\NormalTok{(w) }\CommentTok{#self-normalizing }
  \KeywordTok{return}\NormalTok{(mu)}
\NormalTok{\}}

\NormalTok{mu.estimate <-}\StringTok{ }\KeywordTok{importance_sampling}\NormalTok{(}\DecValTok{10000}\NormalTok{)}
\NormalTok{mu.estimate}
\end{Highlighting}
\end{Shaded}

\begin{verbatim}
## [1] 0.5960878
\end{verbatim}

\begin{Shaded}
\begin{Highlighting}[]
\NormalTok{expected_mean=}\KeywordTok{integrate}\NormalTok{(theta_new_f_g,}\DecValTok{0}\NormalTok{,}\DecValTok{1}\NormalTok{) }
\NormalTok{expected_mean}
\end{Highlighting}
\end{Shaded}

\begin{verbatim}
## 0.5959316 with absolute error < 9.7e-07
\end{verbatim}

The estimated mean is the same as the expected mean, which somewhat
validates the implementation. We observe that this mean is smaller than
with the uniform prior. This can be explained by the shape of the
priors. While the \(Beta(1,1)\) is uniform on \([0,1]\), the
\(Beta(1,5)\)-distribution is much larger for small \(\theta\) and close
to 0 for \(\theta>0.75\). This necessarily leads to fewer samples for
\(\theta\) close to 1, which again skewes the mean downwards.


\end{document}
